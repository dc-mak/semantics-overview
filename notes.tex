\usepackage{proof}
\usepackage{ebproof}

\title{Programming Language Semantics}
\subtitle{An overview of operational, denotational and axiomatic styles of semantics}
\date{11th November, 2019}
\author{Dhruv Makwana}
\institute{%
Ten Minute Tech Net\\%
Engineering, Goldman Sachs}

\graphicspath{{./graphics/}}

\mode<article>{\lecturenumber{1}}

\addtolength{\jot}{\baselineskip}

\begin{document}

\maketitle

\mode*  % ignore text outside frame in presentation mode

% This is your lecture abstract. It should contain a short description of the
% lecture contents, what its goals are, and preferably not exceed a few dozen
% words.\marginnote{Do not forget to use margin notes to add more context to
% your material.}

These notes aim to provide a very brief overview of three styles of formal
semantics used to describe and understand programming langauges, by
defining the syntax and semantics of a small, imperative langauge called `IMP'.
They borrow heavily from~\citet{winskel1993formal} for the structure and
examples.

\section{Syntax of IMP}

\begin{frame}
    \begin{block}{Syntax of IMP}
    \[\begin{array}{rcl}
        a & ::= & n
                \mid X
                \mid a_0 + a_1
                \mid a_0 - a_1
                \mid a_0 \times a_1 \\
                \\
        b & ::= & \mathbf{true}
                \mid \mathbf{false}
                \mid a_0 = a_1
                \mid a_0 \leq a_1
                \mid ¬b
                \mid b_0 \wedge b_1
                \mid b_0 \vee b_1 \\
                \\
        c & ::= & \mathbf{skip}
                \mid X := a
                \mid c_0; c_1
                \mid \mathbf{if}\ b\ \mathbf{then}\ c_0\ \mathbf{else}\ c_1 \\
                && \mathbf{while}\ b\ \mathbf{do}\ c
    \end{array} \]
    \end{block}%{Syntax of IMP}
    \pause
    \begin{enumerate}
        \item No functions (for simplicity's sake).
        \item All terms are "well-typed" by definition.
        \item In code: either tagged-unions or inheritance.
    \end{enumerate}
\end{frame}

\section{Operation Semantics of IMP}

\begin{frame}
    \mode<article>{Operational semantics are an abstract, mathematical specification of an interpreter.\\}
    $\langle a, \sigma \rangle \rightarrow v$ specifies an \emph{evaluation function}, from
    a pair of an arithmetic expression $a$ and a state $\sigma$ to an integer $n$.\\
    \\
    \pause
    \begin{block}{Evaluation of Arithmetic Expressions}
        \begin{gather*}
            \infer{\langle n , \sigma \rangle \rightarrow n}{} \\
            \infer{\langle X , \sigma \rangle \rightarrow \sigma(X)}{} \\
            \infer[n_\mathrm{sum} = n_0 + n_1]{\langle a_0 + a_1 , \sigma \rangle \rightarrow n_\mathrm{sum}}{\langle a_0 , \sigma \rangle \rightarrow n_0 \qquad \langle a_1 , \sigma \rangle \rightarrow n_1}
        \end{gather*}
    \end{block}%{Evaluation of Arithmetic Expressions}
\end{frame}

\begin{frame}<beamer>[standout]
  Thank you.
\end{frame}

\bibliographystyle{plainnat}
\bibliography{references}

\end{document}
